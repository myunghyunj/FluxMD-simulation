\documentclass{article}
\usepackage{libertine}
\usepackage{graphicx}
\usepackage{xcolor}
\usepackage{amsmath}
\usepackage{geometry}
\usepackage{subcaption}
\geometry{letterpaper, vertical, margin=1in}  % ⬅️ landscape 추가됨
\usepackage{caption}
\usepackage{booktabs}
\usepackage{longtable}
\usepackage{array}
\captionsetup[figure]{labelfont={bf}}

\definecolor{customcolor}{HTML}{004191} % KAIST color

\begin{document}

% Title Bar
\noindent
\colorbox{customcolor}{\parbox[t]{\textwidth}{\color{white}\bfseries\fontsize{14pt}{16pt}\selectfont '25 Autumn \textbar \text{ Undergraduate Thesis} \hfill}}
\vspace{.5cm}

% Title
{\flushleft \Large \bfseries FluxMD: A Physics-Based Approach to Binding Site Identification\par}
\vspace{2em}

\begin{abstract}
We present the rigorous mathematical framework underlying FluxMD, a novel computational method that conceptualizes molecular binding sites as energy flux convergence regions in continuous protein surfaces. Unlike traditional structural approaches that discretize binding sites based on geometric features, FluxMD treats molecular recognition as a continuous field phenomenon where binding sites emerge as singularities in the energy flux landscape. We derive the flux differential equation from first principles, validate its mathematical properties, and demonstrate how it naturally captures both enthalpic and entropic contributions to binding.
\end{abstract}

\section{Introduction}

The fundamental insight of FluxMD is that binding sites are not static structural features but dynamic regions where molecular interaction forces converge and dissipate energy. This perspective transforms the continuous protein surface into a differentiable energy flux field $\Phi(\mathbf{r})$, where binding sites manifest as local maxima—energy ``sinkholes'' that attract and stabilize ligands.

\section{Mathematical Foundation}

\subsection{Energy Flux Tensor Field}

Consider a protein-ligand system where the ligand approaches the protein surface along trajectory $\gamma(t)$. At each point in space and time, we define the instantaneous energy flux tensor:

\begin{equation}
\mathbf{T}_i(t) = \sum_{j \in \text{ligand}} E_{ij}(t) \cdot \hat{\mathbf{r}}_{ij}(t)
\end{equation}

where:
\begin{itemize}
    \item $E_{ij}(t)$ is the interaction energy between protein atom $i$ and ligand atom $j$
    \item $\hat{\mathbf{r}}_{ij}(t) = \frac{\mathbf{r}_j - \mathbf{r}_i}{|\mathbf{r}_j - \mathbf{r}_i|}$ is the unit vector from protein to ligand atom
\end{itemize}

\subsection{Interaction Energy Formulation}

The total interaction energy $E_{ij}$ comprises multiple non-covalent contributions:

\begin{equation}
E_{ij} = E_{ij}^{\text{vdW}} + E_{ij}^{\text{elec}} + E_{ij}^{\text{HB}} + E_{ij}^{\pi} + E_{ij}^{\text{solv}}
\end{equation}

Each component is rigorously defined:

\subsubsection{Van der Waals Interactions}
\begin{equation}
E_{ij}^{\text{vdW}} = 4\epsilon_{ij} \left[\left(\frac{\sigma_{ij}}{r_{ij}}\right)^{12} - \left(\frac{\sigma_{ij}}{r_{ij}}\right)^6\right]
\end{equation}

\subsubsection{Hydrogen Bonds}
\begin{equation}
E_{ij}^{\text{HB}} = -\epsilon_{\text{HB}} \left[5\left(\frac{d_0}{r_{ij}}\right)^{12} - 6\left(\frac{d_0}{r_{ij}}\right)^{10}\right] \cdot f(\theta)
\end{equation}
where $f(\theta) = \cos^2(\theta_{\text{DHA}})$ for $\theta_{\text{DHA}} > 120°$, else 0.

\subsubsection{$\pi$-$\pi$ Stacking}
For aromatic ring interactions:
\begin{equation}
E_{ij}^{\pi} = \max\left\{E_{\parallel}(r,\theta), E_{\perp}(r,\theta)\right\} \cdot g(r) \cdot h(\delta)
\end{equation}
where:
\begin{align}
E_{\parallel}(r,\theta) &= -4.0 \exp\left[-\left(\frac{\theta}{30°}\right)^2\right] \\
E_{\perp}(r,\theta) &= -3.5 \exp\left[-\left(\frac{\theta - 90°}{40°}\right)^2\right] \\
g(r) &= \exp\left[-\left(\frac{r - 3.8}{1.5}\right)^2\right] \\
h(\delta) &= \exp\left[-\left(\frac{\delta - 1.5}{2.0}\right)^2\right]
\end{align}

\subsection{Flux Differential Equation}

The energy flux differential for residue $i$ is computed as:

\begin{equation}
\Phi_i = \langle|\mathbf{E}_i|\rangle \cdot C_i \cdot (1 + \tau_i)
\end{equation}

where each term has specific physical meaning:

\subsubsection{Mean Energy Magnitude}
\begin{equation}
\langle|\mathbf{E}_i|\rangle = \frac{1}{N_t} \sum_{t=1}^{N_t} \left\|\sum_{j \in \text{atoms}_i} \mathbf{T}_j(t)\right\|
\end{equation}

This represents the time-averaged magnitude of energy flux through residue $i$.

\subsubsection{Directional Consistency}
\begin{equation}
C_i = \frac{1}{2}\left(1 + \frac{\left\langle\mathbf{T}_i \cdot \hat{\mathbf{T}}_{\text{mean}}\right\rangle}{\langle|\mathbf{T}_i|\rangle}\right)
\end{equation}

where $\hat{\mathbf{T}}_{\text{mean}} = \frac{\langle\mathbf{T}_i\rangle}{|\langle\mathbf{T}_i\rangle|}$. This term quantifies the coherence of energy flux directions.

\subsubsection{Temporal Fluctuation Rate}
\begin{equation}
\tau_i = \sqrt{\frac{1}{N_t-1} \sum_{t=1}^{N_t-1} \left(\frac{d|\mathbf{T}_i|}{dt}\right)^2}
\end{equation}

This captures the dynamic nature of the binding site, with higher fluctuations indicating more flexible regions.

\section{Statistical Validation Framework}

\subsection{Bootstrap Analysis}

To assess statistical significance, we employ bootstrap resampling:

\begin{equation}
\Phi_i^{(b)} = \frac{1}{N_t} \sum_{k=1}^{N_t} \Phi_i(t_{I_k^{(b)}})
\end{equation}

where $I_k^{(b)}$ are randomly sampled trajectory indices with replacement.

\subsection{Confidence Intervals}

The 95\% confidence interval for each residue:
\begin{equation}
\text{CI}_{95\%} = \left[\Phi_i^{(0.025)}, \Phi_i^{(0.975)}\right]
\end{equation}

\subsection{Significance Testing}

The p-value for testing $H_0: \Phi_i = 0$ versus $H_1: \Phi_i > 0$:
\begin{equation}
p_i = \frac{\#\{\Phi_i^{(b)} \leq 0\}}{B}
\end{equation}

where $B$ is the number of bootstrap samples.

\subsection{Effect Size}

Cohen's d for each residue:
\begin{equation}
d_i = \frac{\langle\Phi_i\rangle}{\sigma_{\Phi_i}}
\end{equation}

\section{Physical Interpretation}

\subsection{Energy Flux as a Field}

The flux field $\Phi(\mathbf{r})$ can be interpreted as a scalar field over the protein surface:

\begin{equation}
\Phi(\mathbf{r}) = \int_{\Omega} K(\mathbf{r}, \mathbf{r'}) \Phi_i \, d\mathbf{r'}
\end{equation}

where $K(\mathbf{r}, \mathbf{r'})$ is a smoothing kernel (e.g., Gaussian).

\subsection{Binding Sites as Flux Singularities}

Binding sites emerge where:
\begin{equation}
\nabla^2 \Phi(\mathbf{r}) < 0 \quad \text{and} \quad \Phi(\mathbf{r}) > \Phi_{\text{threshold}}
\end{equation}

These conditions identify local maxima in the flux field that exceed statistical significance.

\subsection{Continuous to Discrete Transition}

The transformation from continuous surface to discrete binding sites occurs through:
\begin{equation}
\mathcal{B} = \{\mathbf{r} : \Phi(\mathbf{r}) > \mu_\Phi + k\sigma_\Phi \text{ and } p(\mathbf{r}) < 0.05\}
\end{equation}

where $k$ is typically 2-3 standard deviations.

\section{Computational Implementation}

\subsection{GPU Acceleration}

The computational complexity is reduced from $O(N_{\text{protein}} \times N_{\text{ligand}})$ to $O(N_{\text{protein}} \log N_{\text{protein}})$ using spatial hashing:

\begin{equation}
h(\mathbf{x}) = \left(\mathbf{x} \cdot \mathbf{p}_1 \oplus \mathbf{x} \cdot \mathbf{p}_2 \oplus \mathbf{x} \cdot \mathbf{p}_3\right) \mod M
\end{equation}

where $\mathbf{p}_i$ are large prime numbers and $M$ is the hash table size.

\subsection{Unified Memory Architecture}

On Apple Silicon, the unified memory allows efficient computation:
\begin{equation}
\text{Speedup} = \frac{T_{\text{CPU}}}{T_{\text{GPU}}} \approx \frac{N}{p} \cdot \frac{1}{1 + \alpha}
\end{equation}

where $p$ is the number of GPU cores and $\alpha$ is the memory transfer overhead (negligible on unified architecture).

\section{Theoretical Properties}

\subsection{Conservation of Energy Flux}

The total energy flux is conserved across the protein surface:
\begin{equation}
\oint_{\partial V} \Phi(\mathbf{r}) \cdot \hat{\mathbf{n}} \, dS = 0
\end{equation}

\subsection{Flux Convergence Theorem}

For a true binding site at position $\mathbf{r}_0$:
\begin{equation}
\lim_{r \to 0} \int_{S_r(\mathbf{r}_0)} \Phi(\mathbf{r}) \, dS = \Phi_{\text{max}}
\end{equation}

where $S_r(\mathbf{r}_0)$ is a sphere of radius $r$ centered at $\mathbf{r}_0$.

\section{Conclusions}

The FluxMD framework provides a mathematically rigorous and physically meaningful approach to identifying molecular binding sites. By treating binding as an energy flux phenomenon rather than a structural feature, we capture the dynamic, continuous nature of molecular recognition. The flux differential equation $\Phi_i = \langle|\mathbf{E}_i|\rangle \cdot C_i \cdot (1 + \tau_i)$ elegantly combines energetic favorability, directional coherence, and dynamic flexibility into a single metric that successfully identifies binding sites as energy flux convergence regions.

\section{Acknowledgments}

This mathematical framework leverages GPU acceleration, particularly the unified memory architecture of Apple Silicon, to make the computations tractable for large biomolecular systems.


\end{document}